\documentclass[a4paper,dvipsnames, 11pt]{amsart} %pdf latex
\usepackage{preamble}

\begin{document}
\maketitle
\begin{notation}
	We employ the following notations. 
	\begin{itemize}
		\item %
			The arrow category of a category $\one{C}$ is denoted by $\one{C}^\ra$.
		\item %
			For each class $\zS$ of morphisms in $\one{C}$,
			we write $\bar\oS$ for the corresponding full subcategory of $\one{C}^\ra$.
		\item %
			Given two morphisms $f,g$, we write $f\perp g$
			if the lifting problem solves uniquely; i.e.,
			the canonical function $\one{C}(\cod(f),\dom(g))\arr\one{C}^\ra(f,g)$ is a bijection.
		\item %
			We write $\Pmor(\one{C})$ for the (large) poset of classes of morphisms in $\one{C}$.
			This is isomorphic to the (large) poset of full subcategories of $\one{C}^\ra$.
		\qedhere %
	\end{itemize}
\end{notation}
\section{Orthogonal factorisation systems}
Let $\one{C}$ be a category.
\subsection{Orthogonality for classes of morphisms}
\begin{definition}
	A class $\zS$ of morphisms is called \emph{reflective} / \emph{coreflective} / \emph{replete} if
	the full subcategory $\bar\oS$ of $\one{C}^\ra$ is reflective, coreflective, and replete respectively.
\end{definition}
\begin{definition}
	Define a monotone function
	\[
		-\perp-
		\colon
		\Pmor(\one{C})^\op
		\times
		\Pmor(\one{C})^\op
		\arr
		\ra
	\]
	by $\zS\perp\zT:=\forall s\in\zS,\ \forall t\in\zT,\ s\perp t$.
	One can easily check $\zS\perp-$ and $-\perp \zS$ preserve limits for any $\zS\in\Pmor(\one{C})$,
	and hence they are representable. We write $\llpo{(-)}\colon\Pmor(\one{C})\adjmap\Pmor(\one{C})^\op\lon\rlpo{(-)}$ for the induced Galois connection.
\end{definition}
\begin{definition}
	An \emph{orthogonal system} $(\zero{E},\zero{M})$ is a pair of classes of morphisms
	satisfying $\llpo{\zero{M}}=\zero{E}$ and $\rlpo{\zero{E}}=\zero{M}$.
\end{definition}
\begin{proposition}
\end{proposition}
\subsection{Wide class of morphisms as subdouble category}
\begin{definition}
	A \emph{(strict) double category} $\dS$ is a category internal to $\Catone$.
	In particular, $\dS$ consists of the following data.
	\[
		\begin{tikzcd}
			\HcatC\dS
			\fibtimes{\bs}{\ts}
			\HcatC\dS
			\ar[r,shift left=3]
			\ar[r,"\fatsemi"description]
			\ar[r,shift right=3]
				&
				\HcatC\dS
				\ar[r,shift left=3,"\bs"]
				\ar[r,shift right=3,"\ts"']
					&
					\Hcat\dS
					\ar[l,"\id"description]
		\end{tikzcd}
	\]
	A \emph{(strict) double functor} $F\colon \dS\arr \dS'$ is a pair of functors
	$\Hcat F\colon\Hcat\dS\arr\Hcat\dS'$ and $\HcatC F\colon\HcatC\dS\arr\Hcat\dS'$ making the obvious diagram commute.
	We write $\SDbl$ for the category of double categories and double functors.

	A \emph{horizontally full subdouble category} of $\dS$ is a double category equipped with a double functor $F$ towards $\dS$
	such that both $\Hcat F$ and $\HcatC F$ are full subcategory-inclusions.
	A horizontally full subdoble category is called \emph{wide} if $\Hcat F$ is an identitity and
	$\HcatC F$ defines a replete subcategory.
	We write $\hfSub(\dS)$ for the poset of horizontally full subdouble categories, and $\whfSub(\dS)$ for its subposet consisting of wide ones.
\end{definition}
\begin{example}
	Finite ordinals $([2],[1],[0])$ form a category object in $\Catone^\op$,
	and hence the internal hom functor $(-_2)^{(-_1)}\colon\Catone^\op\times\Catone\arr\Catone$ induces a functor
	$\Sq\colon\Catone\arr\SDbl$ since $\one{C}^{(-)}\colon\Catone^\op\arr\Catone$ preserves limits for each $\one{C}\in\Catone$.

	For each $\one{C}\in\Catone$, vertical arrows and horizontal arrows
	in $\Sq(\one{C})$ are morphisms in $\one{C}$, while a (unique) cell exists for a square in $\Sq(\one{C})$
	if and only if it is a commutative square in $\one{C}$.
\end{example}
\begin{definition}
	A \emph{wide subcategory} $\one{W}$ of $\one{C}$ is a replete subcategory of $\one{C}$ whose inclusion $\one{W}\arr[hook]\one{C}$
	is (essentially) surjective.
\end{definition}
\begin{proposition}[Subcategory axiom]
	There is an isomorphism of posets
	\[
		\Sub(\one{C})\cong\hfSub(\Sq(\one{C}))
		\text{,}
	\]
	where $\Sub(\one{C})$ is the poset of subcategories of $\one{C}$.
	Moreover, it restricts to another isomorphism
	\[
		\wSub(\one{C})\cong\whfSub(\Sq(\one{C}))
		\text{,}
	\]
	where $\wSub(\one{C})$ is the poset of wide subcategories.
\end{proposition}
\begin{proof}
	A subcategory of $\one{C}$ and a horizontally full subdouble category of $\Sq(\one{C})$
	specify the same data:
	a class of arrows $\zS$ such that $\zS$ is closed under composition, and
	for each $f\in\zS$, the identities on $\dom(f)$ and $\cod(f)$ are in $\zS$.
	The subcategory and the subdouble category are wide simultaneously
	if and only if $\zS$ contains all isomorphisms in $\one{C}$.
\end{proof}
\begin{notation}
	We say a class $\zS$ of morphisms is \emph{wide} if it is closed under composition and contains isomorphisms.
	We write $\oS$ and $\dS$ for the corresponding wide subcategory and wide horizontally full subdouble category.
	Note that $\bar\oS$ coincide with $\HcatC\dS$.
\end{notation}

\subsection{Functorial factorisation}
\begin{definition}
	A \emph{functorial factorisation} on $\one{C}$
	is a factorisation of the canonical natural transformation
	$\dom\arr[Rightarrow]\cod\colon\one{C}^\ra\arr\one{C}$.
	In other words, a functorial factorisation $F=(\sfE,\bar\sfL,\bar\sfR)$
	consists of the following data.
	\begin{itemize}
		\item %
			A functor $\sfE\colon\one{C}^\ra\arr\one{C}$.
		\item %
			Two natural transformations
			$\dom\arr"\bar\sfL"[Rightarrow] \sfE\arr"\bar\sfR"[Rightarrow]\cod$
			whose composite is the canonical natural transformation $\dom\arr[Rightarrow]\cod$.
	\end{itemize}
	We write $\sfL\colon\one{C}^\ra\arr\one{C}^\ra$ 
	and $\sfR\colon\one{C}^\ra\arr\one{C}^\ra$ for the functors
	obtained from
	$\bar\sfL$ and $\bar\sfR$ respectively
	by
	the natural isomorphism
	$\left[\ra,\left[\one{C}^\ra,\one{C}\right]\right]\cong \left[\one{C}^\ra,\one{C}^\ra\right]$.
\end{definition}

\bibliographystyle{halpha-abbrv}
\bibliography{bibliography}
\end{document}

	Let $\one{S}$ be a subcategory.
	Define a subdouble category $\dS$ of $\Sq(\one{C})$ as follows.
	\begin{itemize}
		\item %
			$\HcatC(\dS)$
			is defined by the bo-ff factorisation of the functor
			$\one{S}^\ra\arr\one{C}^\ra$
			induced from the inclusion $\one{S}\arr[tail]\one{C}$.
		\item %
			The underlying category $\Hcat(\dS)$
			is obtained by the the bo-ff factorisation of
			\[
				\HcatC(\dS)\arr[hook]\one{C}^\ra\arr\one{C}
			\]
			where the latter functor is either of the codomain or the domain.
	\end{itemize}
