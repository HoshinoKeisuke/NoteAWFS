\documentclass[a4paper,dvipsnames, 11pt]{amsart} %pdf latex
\usepackage{preamble}

\begin{document}
\maketitle
\begin{notation}
	We employ the following notations. 
	\begin{itemize}
		\item %
	\end{itemize}
\end{notation}
\section{Orthogonal factorisation systems}
Let $\one{C}$ be a category.
\begin{definition}
	A \emph{functorial factorisation} on $\one{C}$
	is a factorisation of the canonical natural transformation
	$\dom\arr[Rightarrow]\cod\colon\one{C}^\one{2}\arr\one{C}$.
	In other words, a functorial factorisation $F=(E,\eta,\varepsilon)$
	consists of the following data.
	\begin{itemize}
		\item %
			A functor $E\colon\one{C}^\one{2}\arr\one{C}$.
		\item %
			Two natural transformations
			$\dom\arr"\eta"[Rightarrow] E\arr"\varepsilon"[Rightarrow]\cod$
			whose composite is the canonical natural transformation $\dom\arr[Rightarrow]\cod$.
		\qedhere %
	\end{itemize}
\end{definition}

\begin{definition}
	A \emph{(strict) double category} $\dS$ is a category internal to $\Catone$.
	In particular, $\dS$ consists of the following data.
	\[
		\begin{tikzcd}
			\HcatC\dS
			\fibtimes{\bs}{\ts}
			\HcatC\dS
			\ar[r,shift left=3]
			\ar[r,"\fatsemi"description]
			\ar[r,shift right=3]
				&
				\HcatC\dS
				\ar[r,shift left=3,"\bs"]
				\ar[r,shift right=3,"\ts"']
					&
					\Hcat\dS
					\ar[l,"\id"description]
		\end{tikzcd}
	\]
	A \emph{(strict) double functor} $F\colon \dS\arr \dS'$ is a pair of functors
	$\Hcat F\colon\Hcat\dS\arr\Hcat\dS'$ and $\HcatC F\colon\HcatC\dS\arr\Hcat\dS'$ making the obvious diagram commute.
	We write $\SDbl$ for the category of double categories and double functors.

	A \emph{horizontally full subdouble category} of $\dS$ is a double category equipped with a double functor $F$ towards $\dS$
	such that both $\Hcat F$ and $\HcatC F$ are full subcategory inclusions.
	A horizontally full subdoble category is called \emph{wide} if $\Hcat F$ is an identitity and
	$\HcatC F$ is a replete subcategory inclusion.
	We write $\HFSub(\dS)$ for the poset of horizontally full subdouble categories, and $\WHFSub(\dS)$ for its subposet consisting of wide ones.
\end{definition}
\begin{example}
	(Nerves of) simplices $([2],[1],[0])$ form a category object in $\Catone^\op$,
	and hence the internal hom functor $(-_2)^{(-_1)}\colon\Catone^\op\times\Catone\arr\Catone$ induces a functor
	$\Sq\colon\Catone\arr\SDbl$ since $\one{C}^{(-)}\colon\Catone^\op\arr\Catone$ preserves limits for each $\one{C}\in\Catone$.

	For each $\one{C}\in\Catone$, vertical arrows and horizontal arrows
	in $\Sq(\one{C})$ are morphisms in $\one{C}$, while a (unique) cell exists for a square in $\Sq(\one{C})$
	if and only if it is a commutative square in $\one{C}$.
\end{example}
\begin{proposition}[Subcategory axiom]
	There is an isomorphism of posets
	\[
		\Sub(\one{C})\cong\HFSub(\Sq(\one{C}))
		\text{,}
	\]
	where $\Sub(\one{C})$ is the poset of subcategories of $\one{C}$.
	Moreover, it restricts to another isomorphism
	\[
		\WSub(\one{C})\cong\WHFSub(\Sq(\one{C}))
		\text{,}
	\]
	where $\WSub(\one{C})$ is the poset of wide subcategories.
\end{proposition}
\begin{definition}
	A \emph{wide subcategory} $\one{W}$ of $\one{C}$ is a replete subcategory of $\one{C}$ whose inclusion $\one{W}\arr[hook]\one{C}$
	is (essentially) surjective.
\end{definition}
\bibliographystyle{halpha-abbrv}
\bibliography{bibliography}
\end{document}
